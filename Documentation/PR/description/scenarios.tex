% !TEX root = knauss-vissuelizer.tex
\section{Example Scenarios}
In this section we describe examples that highlight the functionality of the \viss\ tool.

\subsection{Where are the hotspots in a set of issues?}
\begin{figure}[b]
\includegraphics[width=\columnwidth]{img/example-trajectory}
\caption{Example of a requirements discussion's clarification trajectory}
\label{fig:example-trajectory}
\end{figure}
Often, a clear understanding of requirements only evolves during the development of software.
This is especially true (but not limited to) agile software projects, where managers decide to frame only rudimentary requirements and refine the details on the go.
For a manager, it is important to know when problematic requirements surface, because they can have a serious impact on the project.
\viss\ helps managers in this scenario as follows:
\begin{enumerate}
\item The manager loads a set of requirements (e.g. user stories for the current iteration).
\item \viss\ automatically analyzes the discussion events that are related to these requirements and that are available in online communication.  
\item \viss\ creates a set of \emph{clarification trajectories} (c.f. Figure \ref{fig:example-trajectory}), one for each requirement. 
%Discussion events that are concerned with clarifying requirements are depicted by red rectangles below the timeline, other comments are depicted by blue rectangles above the timeline.
\item \viss\ also displays suggestive pattern names for distinctive trajectories (e.g. textbook-example, back-to-draft, procrastination).
\item The manager scrolls through the requirements and associated trajectories and decides based on this rich information where to invest more resources.
\end{enumerate}
Typically, there is a number of requirements without pathological findings, e.g. user stories with some clarification in the beginning and other communication events later on that show progress. 
But there are also suspicious trajectories, e.g. a large amount of clarification late in the iteration, perhaps even after the issue seemed to be solved, or no clarification at all, even though the requirement seems to be complex.

\subsection{Are there any communication breakdowns?}
After identifying those hotspots, the manager most likely wants to continue with a closer investigation. 
Often, he or she will investigate who participates in a discussion of a requirement and who is not.
\viss\ helps managers in this scenario by creating social networks for a requirement or a set of requirements on the fly.
More over, \viss\ also integrates information of the automatic analysis of online communication into these social networks, i.e. showing for each actor the percentage of clarification and other communication.
Figure \ref{fig:example-sn} shows an example of an social network for the requirement presented in Figure \ref{fig:example-trajectory}. 
\begin{figure}
\includegraphics[width=\columnwidth]{img/example-sn}
\caption{Example of a requirements discussion's social network}
\label{fig:example-sn}
\end{figure}
The developers are presented as nodes (here: anonymized), and connections between nodes are weighted by the amount of communication both developers share about a given requirement in a specific time interval. 
In the example above, the manager might conclude that there is no single person who is coordinating the analysis and work around this requirement, because there is no actor who participates in all relevant time intervals.
A suitable action might be to assign this responsibility to a more experienced developer.

\subsection{Who is knowledgeable about a given topic?}

Integrating the right persons in the loop for an important feature is a crucial ability for managers.
To support managers in this task, \viss\ distinguishes between two types of knowledge: domain knowledge that shows in communication events related to clarification and technical knowledge that shows in other communication events.
In order to leverage the power of \viss\ for this task, the manager selects a number of requirements that are related to a given topic. 
\viss\ integrates the social networks of these requirements discussions in a single large network (see Figure \ref{fig:example-sn-large}).
\begin{figure}
\includegraphics[width=\columnwidth]{img/example-sn-large}
\caption{Example of a social network for a set of requirements discussions}
\label{fig:example-sn-large}
\end{figure}
The manager looks for candidates with a balanced percentage of clarification and other communication.
Candidates should also have at least a medium centrality, but actors with high centrality might already have a very high workload.
